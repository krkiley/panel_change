% Options for packages loaded elsewhere
\PassOptionsToPackage{unicode}{hyperref}
\PassOptionsToPackage{hyphens}{url}
%
\documentclass[
]{article}
\usepackage{amsmath,amssymb}
\usepackage{lmodern}
\usepackage{ifxetex,ifluatex}
\ifnum 0\ifxetex 1\fi\ifluatex 1\fi=0 % if pdftex
  \usepackage[T1]{fontenc}
  \usepackage[utf8]{inputenc}
  \usepackage{textcomp} % provide euro and other symbols
\else % if luatex or xetex
  \usepackage{unicode-math}
  \defaultfontfeatures{Scale=MatchLowercase}
  \defaultfontfeatures[\rmfamily]{Ligatures=TeX,Scale=1}
\fi
% Use upquote if available, for straight quotes in verbatim environments
\IfFileExists{upquote.sty}{\usepackage{upquote}}{}
\IfFileExists{microtype.sty}{% use microtype if available
  \usepackage[]{microtype}
  \UseMicrotypeSet[protrusion]{basicmath} % disable protrusion for tt fonts
}{}
\usepackage{xcolor}
\IfFileExists{xurl.sty}{\usepackage{xurl}}{} % add URL line breaks if available
\IfFileExists{bookmark.sty}{\usepackage{bookmark}}{\usepackage{hyperref}}
\hypersetup{
  pdftitle={Correction: Measuring Stability and Change in Personal Culture Using Panel Data},
  pdfauthor={Kevin Kiley; Stephen Vaisey},
  hidelinks,
  pdfcreator={LaTeX via pandoc}}
\urlstyle{same} % disable monospaced font for URLs
\usepackage[margin=1in]{geometry}
\usepackage{graphicx}
\makeatletter
\def\maxwidth{\ifdim\Gin@nat@width>\linewidth\linewidth\else\Gin@nat@width\fi}
\def\maxheight{\ifdim\Gin@nat@height>\textheight\textheight\else\Gin@nat@height\fi}
\makeatother
% Scale images if necessary, so that they will not overflow the page
% margins by default, and it is still possible to overwrite the defaults
% using explicit options in \includegraphics[width, height, ...]{}
\setkeys{Gin}{width=\maxwidth,height=\maxheight,keepaspectratio}
% Set default figure placement to htbp
\makeatletter
\def\fps@figure{htbp}
\makeatother
\setlength{\emergencystretch}{3em} % prevent overfull lines
\providecommand{\tightlist}{%
  \setlength{\itemsep}{0pt}\setlength{\parskip}{0pt}}
\setcounter{secnumdepth}{5}
\usepackage{setspace}
\doublespacing
\ifluatex
  \usepackage{selnolig}  % disable illegal ligatures
\fi
\newlength{\cslhangindent}
\setlength{\cslhangindent}{1.5em}
\newlength{\csllabelwidth}
\setlength{\csllabelwidth}{3em}
\newenvironment{CSLReferences}[2] % #1 hanging-ident, #2 entry spacing
 {% don't indent paragraphs
  \setlength{\parindent}{0pt}
  % turn on hanging indent if param 1 is 1
  \ifodd #1 \everypar{\setlength{\hangindent}{\cslhangindent}}\ignorespaces\fi
  % set entry spacing
  \ifnum #2 > 0
  \setlength{\parskip}{#2\baselineskip}
  \fi
 }%
 {}
\usepackage{calc}
\newcommand{\CSLBlock}[1]{#1\hfill\break}
\newcommand{\CSLLeftMargin}[1]{\parbox[t]{\csllabelwidth}{#1}}
\newcommand{\CSLRightInline}[1]{\parbox[t]{\linewidth - \csllabelwidth}{#1}\break}
\newcommand{\CSLIndent}[1]{\hspace{\cslhangindent}#1}

\title{Correction: Measuring Stability and Change in Personal Culture
Using Panel Data}
\author{Kevin Kiley \and Stephen Vaisey}
\date{1/3/2022}

\begin{document}
\maketitle

We recently discovered that an error in the R source code for a function
we used in the original analysis for our 2020 ASR paper, ``Measuring
Stability and Change in Personal Culture Using Panel Data,'' led to
erroneous results, and we would like to correct the record. Correcting
this bug in the function, in which survey weights were incorrectly
applied to residuals twice (rather than once) in calculating the log
likelihood of non-linear least squares models, changes how we classify
some items as showing either ``settled dispositions'' or ``active
updating'' throughout the paper. It does not affect the overall pattern
of results or our substantive interpretations of those results. We stand
by the general claims we made in the paper, which we reproduced with
alternative methods elsewhere (Vaisey and Kiley 2021).

In our original paper, we estimated two non-linear least squares models
-- one that modeled changes within people over time as random departures
from a settled baseline, which we called a ``settled dispositions
model,'' and one that modeled changes within people over time as durable
updates, which we called an ``active updating model'' -- for 183 items,
including several composite scales, from the General Social Survey's
rotating three-wave panels. We then compared the fit of these two models
using the Bayesian Information Criterion to evaluate which model better
fit the observed data. We found the settled dispositions model was
preferred for 75 questions, meaning there was no evidence of durable
change for these items. We also found that even for questions where
there was evidence of durable change, most change was not durable.

We conducted this analysis using R version 3.5.3. In versions of R prior
to 4.0.3, there were two errors in the way R calculated the log
likelihood of non-linear least squares models, which resulted in an
incorrect calculation of the Bayesian Information Criterion. This led to
errors in how we judged the two models against each other.

In versions of R prior to 4.0.3, the log likelihood of non-linear least
squares models was calculated as:

\[ - \frac{N}{2}(log(2\pi) + 1 - log(N) - \sum_{i=1}^N{log(w) + log(\sum_{i=1}^N{[w^2*(y - \hat{y})]^2})}) \]

where \(N\) is the number of observations, \(w\) are survey weights,
\(y\) is the observed value of the outcome, and \(\hat{y}\) is the
predicted value of the outcome. The key issue for our analysis was that
the formula applied survey weights to residuals twice (\(w^2\) instead
of \(w\)). The formula also did not divide the sum of the log of the
weights by N. This latter issue produced log likelihood values that were
too large, but because this term was constant across both models, it did
not affect comparisons.

The corrected calculation of the log likelihood for non-linear least
squares models, implemented in version 4.0.3,\footnote{Documentation of
  the error can be found:
  \url{https://bugs.r-project.org/show_bug.cgi?id=16100\#c3}} is:

\[ - \frac{N}{2}(log(2\pi) + 1 - log(N) - \frac{\sum_{i=1}^N{log(w)}}{N} + log(\sum_{i=1}^N{[w*(y - \hat{y})]^2})) \]

Because the error lies in how survey weights were applied to the
calculation of the BIC, there is no clear pattern in how results changed
when the correct formula was applied. Re-running our analyses with the
corrected calculation produces slightly different results than what we
presented in the published paper. Seventeen variables where we found a
better fit for the stable disposition model prefer the active updating
model under the corrected calculation. Fourteen variables where we found
that the active updating model fit the data better prefer the settled
dispositions model under the corrected calculation. This resulted in a
net change of three more variables preferring the active updating model
to the settled dispositions model (75, compared to 72 in the original
analysis). The revised calculation also found fewer variables with a
statistically significant difference between the \(\phi\) estimates for
people 30 and under and people over 30 (9 instead of 22), but the
overall pattern of younger people showing more active updating is still
consistent.

The items we labeled as ``settled dispositions'' under the incorrect
calculation that are ``active updating'' under the corrected calculation
are: abnomore, confinan, conjudge, contv, courts, fund, hapmar,
misanthropy, neverrelact, polattak, reborn, socommun, socrel, spkath,
spkrac, teensex, and xmovie.

The items we labeled as ``active updating'' under the incorrect
calculation that are ``settled dispositions'' under the corrected
calculation are: aged, colrac, divlaw, getahead, marwht, meovrwrk,
pillok, polhitok, racdif2, socfrend, suicide1, trust, uswary, and
workblks.

While fixing this calculation changes how these variables are
classified, the erroneous calculation was unrelated to our estimates of
\(\phi\) or \(\beta\), and the overall substantive conclusions of our
paper, which we reproduced with different methods elsewhere (Vaisey and
Kiley 2021), remain the same.

Corrected versions of Figures 2, 3, 4, 5, and A1-A4 are included below.
Figures 1 and 6 were not affected by the change. Code to reproduce the
analyses, including the old and corrected results, can be found at:
\url{https://github.com/krkiley/panel_change}.

We thank Hugo Mercier for bringing the discrepancy to our attention. We
thank Sebastian Meyer for patching the function in R and for explaining
the change in personal communications.

\hypertarget{figures}{%
\section{Figures}\label{figures}}

\begin{figure}
\centering
\includegraphics{correction_files/figure-latex/phihist-1.pdf}
\caption{Distribution of phi estimates and probabilities that items show
evidence of active updating.}
\end{figure}

\begin{figure}
\centering
\includegraphics{correction_files/figure-latex/agegroupcomparison-1.pdf}
\caption{Comparison of phi estimates for individuals over and equal to
or less than 30 years old, with 95 percent confidence intervals.}
\end{figure}

\begin{figure}
\centering
\includegraphics{correction_files/figure-latex/phibetascatter-1.pdf}
\caption{Distribution of beta and phi estimates for GSS items, by
whether model prefers phi = .5 constraint.}
\end{figure}

\begin{figure}
\centering
\includegraphics{correction_files/figure-latex/summary-1.pdf}
\caption{Summary of phi estimates for all items, by topical group.
Numbers in parentheses indicate the number of items in each topical
group.}
\end{figure}

\begin{figure}
\centering
\includegraphics{correction_files/figure-latex/religphi-1.pdf}
\caption{Phi estimates for items about religious activity and beliefs,
social life, subjective SES, and suicide.}
\end{figure}

\begin{figure}
\centering
\includegraphics{correction_files/figure-latex/poliphi-1.pdf}
\caption{Phi estimates for items about guns, law, crime and policing;
politics and government; and public spending. The item `Police can hit
citizens,' which has phi = .43, has been removed for ease of viewing}
\end{figure}

\begin{figure}
\centering
\includegraphics{correction_files/figure-latex/civlibphi-1.pdf}
\caption{Phi estimates for items about civil liberties, confidence in
leadership, health, morale, and social trust.}
\end{figure}

\begin{figure}
\centering
\includegraphics{correction_files/figure-latex/racgenphi-1.pdf}
\caption{Phi estimates for items about race, gender, sex, sexuality, and
abortion.}
\end{figure}

\hypertarget{references}{%
\section*{References}\label{references}}
\addcontentsline{toc}{section}{References}

\hypertarget{refs}{}
\begin{CSLReferences}{1}{0}
\leavevmode\hypertarget{ref-vaisey2021}{}%
Vaisey, Stephen, and Kevin Kiley. 2021. {``A Model-Based Method for
Detecting Persistent Cultural Change Using Panel Data.''}
\emph{Sociological Science} 8 (5): 83--95.
\url{https://doi.org/10.15195/v8.a5}.

\end{CSLReferences}

\end{document}
